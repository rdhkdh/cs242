% Ridhiman Kaur Dhindsa, Roll no.210101088
% Programming Assignment 4, Ques 2(i)
\documentclass[a4paper]{article}
\usepackage{mathtools}
\usepackage{amsmath}
\usepackage[hidelinks,colorlinks=true,linkcolor=blue,citecolor=blue]{hyperref}

\begin{document}

\begin{center}
\begin{large}
	Hello World!\\
\end{large}
\end{center}

\begin{enumerate}

\begin{large}
\item[1] \textbf{Getting Started}
\end{large}\\
\\
\textbf{Hello World!} Today I am learning \LaTeX\ . \LaTeX\ is a great program for writing math. I can write in line math such as $a^2+b^2=c^2$. I can also give equations their own space:
\begin{equation}
\gamma^2+\theta^2=\omega^2 \label{eq:a}
\end{equation}
``Maxwell's equation'' are named for James Clark Maxwell and are as follows:

\begin{flalign}
\vec{\nabla}\cdot\vec{E}=\frac{\rho}{\epsilon_{0}}&&\text{Gauss's Law}\label{eq:b}\\
\vec{\nabla}\cdot\vec{B}=0&&\text{Gauss's Law for Magnetism}\label{eq:c}\\
\vec{\nabla}\times\vec{E}=-\frac{\partial \vec{B}}{\partial t}&&\text{Faraday's Law of Induction}\label{eq:d}\\
\vec{\nabla}\times\vec{B}=\mu_{0}\left(\epsilon_{0}\frac{\partial \vec{E}}{\partial t}+\vec{J}\right)&&\text{Ampere's Circuital Law}\label{eq:e}
\end{flalign}

Equations \eqref{eq:b},\eqref{eq:c},\eqref{eq:d} and \eqref{eq:e} are some of the most important in Physics.

\begin{large}
\item[2] \textbf{What about Matrix Equations?}
\end{large}\\
\\
\[
\begin{pmatrix}
a_{11} & a_{12} & \cdots & a_{1n}\\
a_{21} & a_{22} & \cdots & a_{2n}\\
\vdots & \vdots & \ddots & \vdots \\
a_{n1} & a_{n2} & \cdots & a_{nn}\\
\end{pmatrix}
\begin{bmatrix}
v_{1}\\
v_{2}\\
\vdots \\
v_{n}\\
\end{bmatrix}
=
\begin{matrix}
w_{1} \\
w_{2} \\
\vdots \\
w_{n} \\
\end{matrix}
\]

\end{enumerate}


\end{document}